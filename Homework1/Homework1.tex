\documentclass[11pt]{article}
\usepackage[utf8]{inputenc}
\usepackage[english]{babel}
\usepackage{graphicx}
\usepackage{array}
\usepackage{amsmath}
\usepackage{mathtools}
\usepackage{algorithmicx}
\usepackage{algpseudocode}
\renewcommand{\baselinestretch}{1.0}
\usepackage[letterpaper, margin=0.75in]{geometry}

\begin{document}
	\title{Homework1 for EECS 340}
	\author{Yu Mi}
	\maketitle
\section{Warm-up: Big-Oh and Counting Primitive Operations}
\emph{Show your work} on the following questions. Use the limit-based definitions of asymptotic notation on the ``Big-Oh Cheat Sheet'' on Canvas wherever applicable.
\subsection{Solve R-1.20, R-1.22, and R-1.23 in the text}
\subsubsection{R-1.20}
	Show that $(n+1)^5$ is $O(n^5)$.

	\emph{Proof:} Let $f(n)=(n+1)^5$ and $g(n)=n^5$ so that
	\begin{equation*}
	\begin{multlined}
		\lim_{n\to\infty} \frac{f(n)}{g(n)}=\lim_{n\to\infty} \frac{(n+1)^5}{n^5}=\lim_{n\to\infty} \frac{n^5+5n^4+10n^3+10n^2+5n+1}{n^5}\\
		=1+\lim_{n\to\infty}\frac{5}{n}+\frac{10}{n^2}+\frac{10}{n^3}+\frac{5}{n^4}+\frac{1}{n^5}=1
	\end{multlined}
	\end{equation*}

	Since $0\leq1<\infty$, $(n+1)^5$ is $O(n^5)$.
\subsubsection{R-1.22}
	Show that $n$ is $o(n\log n)$.
	
	\emph{Proof:} Let $f(n)=n$ and $g(n)=n\log n$ so that
	\begin{equation*}
		\lim_{n\to\infty} \frac{f(n)}{g(n)}=\lim_{n\to\infty} \frac{n}{n\log n}=\lim_{n\to\infty} \frac{1}{\log n}=0
	\end{equation*}
	
	Since $0=0$, $n$ is $o(n\log n)$.
\subsubsection{R-1.23}
	Show that $n^2$ is $\omega(n)$.
	
	\emph{Proof:} Let $f(n)=n^2$ and $g(n)=n$ so that
	\begin{equation*}
		\lim_{n\to\infty} \frac{f(n)}{g(n)}=\lim_{n\to\infty} \frac{n^2}{n}=\lim_{n\to\infty} n=\infty
	\end{equation*}
	
	Since $\infty = \infty$, $n^2$ is $\omega(n)$.
\subsection{Intuitively, $2^x\in O(3^x)$, since $3^x$ grows faster. Is $3^x\in O(2^x)$?}
	\emph{Answer:} No, proof as follows:
	
	Let $f(x)=3^x$ and $g(x)=2^x$ so that
	\begin{equation*}
		\lim_{x\to\infty} \frac{f(x)}{g(x)}=\lim_{x\to\infty} \frac{3^x}{2^x}=\lim_{x\to\infty} (\frac{3}{2})^x=\infty
	\end{equation*}
	
	Since $\infty = \infty$, so that $3^x$ is $\omega(2^x)$, $3^x\not\in O(2^x)$
\subsection{Intuitively, $\log_3(x)\in O(\log_2(x))$. Is $\log_2(x)\in O(\log_3(x))$?}
	\emph{Answer:} Yes, proof as follows:
	
	Let $f(x)=\log_2(x)$ and $g(x)=\log_3(x)$ so that
	\begin{equation*}
		\lim_{x\to\infty} \frac{\log_2(x)}{\log_3(x)} = \lim_{x\to\infty} \frac{\frac{\ln x}{\ln 2}}{\frac{\ln x}{\ln 3}} = \lim_{x\to\infty} \frac{\ln 3}{\ln 2} = \log_2(3)
	\end{equation*}
	
	Since $0\leq \log_2(3)<\infty$, $\log_2(x)$ is $O(\log_3(x))$.
\subsection{Use summations to derive tight asymptotic bounds $(\Theta (-))$ on the runtime of each algorithm}
\subsubsection{R-1.12}
	\emph{Answer:}
	First, we need to rewrite this algorithm into \emph{while} loop:
	
	\textbf{Algorithm} Loop2($n$):
	
	\begin{algorithmic}
		\State $p \gets 1$ \Comment $1$ unit of time
		\State $i \gets 1$ \Comment $1$ unit of time
		\While {$i \leq 2n$} \Comment $2n+1$ units of time
			\State $p \gets p \times i$ \Comment $2\times 2n$ units of time
			\State $i \gets i + 1$ \Comment $2\times 2n$ units of time
		\EndWhile
	\end{algorithmic}
	
	As is described in the comments of the algorithm, the run time of this algorithm should be $\Theta(10n+3)$ units of time. So that this algorithm is $\Theta(n)$
\subsubsection{R-1.14}
	\emph{Answer:}
	First, we need to rewrite this algorithm into \emph{while} loop:
	
	\textbf{Algorithm} Loop4($n$):
	
	\begin{algorithmic}
		\State $s \gets 0$ \Comment $1$ unit of time
		\State $i \gets 1$ \Comment $1$ unit of time
		\While {$i \leq 2n$} \Comment $2n+1$ units of time
			\State $j \gets 1$ \Comment $2n$ units of time
				\While {$j \leq i$} \Comment $(i+1) \times 2n$ units of time
					\State $s \gets s+i$ \Comment $2\times i \times 2n$ units of time
					\State $j \gets j+1$ \Comment $2\times i \times 2n$ units of time
				\EndWhile
			\State $i \gets i+1$ \Comment $2n$ units of time
		\EndWhile
	\end{algorithmic}
	
	To calculate the time cost of the inner loop, we need to focus on the value of $i$ which changes with the outer loop. To make calculate easy to understand, we define $C_1$ as the actual units of time the outer loop will cost and $C_2$ as the actual units of time the inner loop will cost. So that:
	\begin{equation*}
		C_1=1+1+2n+1+2n+2n = 6n+3 
	\end{equation*}
	\begin{equation*}
		C_2=2n\sum_{i=1}^{2n} (i+1+2i+2i)=10n^2+7n
	\end{equation*}
	To sum up, the total units of time this algorithm will cost should be $time_{total} = C_1 + C_2 = 10n^2+13n+3$. So that this algorithm is $\Theta(n^2)$
\subsubsection{R-1.15}
	\emph{Answer:}
	First, we need to rewrite this algorithm into \emph{while} loop:
	
	\textbf{Algorithm} Loop5($n$):
	
	\begin{algorithmic}
		\State $s\gets0$ \Comment $1$ unit of time
		\State $i\gets1$ \Comment $1$ unit of time
		\While{$i \leq n^2$} \Comment $n^2+1$ units of time
			\State $j \gets 1$ \Comment $n^2$ units of time
			\While{$j \leq i$} \Comment $(i+1) \times n^2$ units of time
				\State $s\gets s+i$ \Comment $2\times i \times n^2$ units of time
				\State $j\gets j+1$ \Comment $2\times i \times n^2$ units of time
			\EndWhile
			\State $i\gets i+1$ \Comment $n^2$ units of time
		\EndWhile
	\end{algorithmic}
	
	To calculate the time cost of the inner loop, we need to focus on the value of $i$ which changes with the outer loop. To make calculate easy to understand, we define $C_1$ as the actual units of time the outer loop will cost and $C_2$ as the actual units of time the inner loop will cost. So that:
	\begin{equation*}
		C_1=1+1+n^2+1+n^2+n^2 = 3n^2+3
	\end{equation*}
	\begin{equation*}
		C_2=2n\sum_{i=1}^{n^2} (i+1+2i+2i)=\frac{5}{2} n^4 + \frac{7}{2} n^2
	\end{equation*}
	To sum up, the total units of time this algorithm will cost should be $time_{total} = C_1 + C_2 = \frac{5}{2}n^4+\frac{13}{2}n^2+3$. So that this algorithm is $\Theta(n^4)$
\subsection{Explain why it is reasonable to ignore the overhead of a ranged \emph{for} loop when you derived the tight asymptotic runtime bounds in the previous question.}
	\emph{Answer:} When calculating the runtime bounds of a loop, the overhead of such loop will cost a constant time of computation, while the loop body will cost $n$ times more than the overhead. As $n$ grows big enough, the overhead is always significantly smaller than the loop body. Thus, when we are deriving the asymptotic runtime bounds, we can ignore the overhead of a ranged loop.
\end{document}